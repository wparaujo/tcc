\chapter{Introdução}

\section{Contextualização}
Os exames de biópsia são responsáveis de 60\% a 70\% dos diagnósticos patológicos \cite{auguste2015mobile}. Esse exame consiste na retirada de fragmentos teciduais para a observação da lâmina pelo médico patologista.

Com o aumento da capacidade computacional ao longo dos últimos anos e o avanço de estudos em visão computacional e inteligência artificial, surgiu a possibilidade do uso dos recursos tecnológicos, tanto em hardware quanto em software, para o conceito de patologia computacional, que é a utilização dos dados médicos, como imagens, para o auxílio em diagnóstico patológico, compartilhamento de informações entre o corpo médico e ensino \cite{louis2014computational}.

Para que se possa digitalizar lâminas histopatológicas, é necessário que haja equipamentos WSI, responsáveis por capturar fragmentos das imagens da lâmina para que se possa construí-las por completo, dado que a ampliação microscópica capta apenas uma determinada área dependente de sua magnitude de observação. Os WSI possuem duas etapas de aquisição de imagem, a de hardware para captação da imagem e o software responsável por visualizar as imagens, manipulá-las e, por vezes, terem ferramentas de visão computacional e inteligência artificial. O valor de um WSI pode variar de  \$ 30 000 à \$ 300 000 dólares, fazendo com que seja dfícil sua implementação devido ao alto custo de aquisição \cite{auguste2015mobile}.

% Falar sobre Image Stitching, algoritmos de scan, etc.

\section{Questão de Pesquisa}

\section{Justificativa}

\section{Objetivo Geral}

Este trabalho tem como objetivo desenvolver uma ferramenta de baixo custo para a digitalização de lâminas histopatológicas, utilizando um smartphone acoplado a um microscópio por um suporte apropriado.

\subsection{Objetivos Específicos}
A fim de atingir o objetivo geral, foram determinados os seguintes objetivos específicos:

\begin{enumerate}
    \item Desenvolver um aplicativo iOS para captação dos framentos da imagem;
    \item Estabelecer um servidor em nuvem para o processamento de imagem responsavél pela composição;
    \item Identificar algoritmo para a composição da imagem;
    \item Estabelecer requisitos para o software de visualizador das lâminas histopatológicas digitalizadas;
    \item Desenvolver o software visualizador;
\end{enumerate}

\section{Organização do Documento}





